\documentclass[12pt,a4paper]{article}
\usepackage[utf8]{inputenc}
\usepackage[T1]{fontenc}
\usepackage{amsmath,amssymb,amsfonts}
\usepackage[left=2cm,right=2cm,top=2cm,bottom=2cm]{geometry}
\usepackage{hyperref}

\title{Permutações em Espaços Densos}
\author{}
\date{}

\begin{document}

\maketitle

\section{Permutações em espaços densos}

Dado $(M, +, \leq)$ um monóide de adição, ordenado e denso de forma que para quaisquer $a,b \in M$ com $a<b$, existe $c \in M$ tal que $a < c < b$, definimos a operação $p$ de permutação de par de intervalos para espaços densos que é uma bijeção $p_{\alpha,\beta}: M \to M$ onde $x \in M$ e $\alpha,\beta \in \mathbb{N}$ e escolhemos 2 intervalos arbitrários $I_{\alpha}$ e $I_{\beta}$ de forma que:

\begin{equation}
 p_{\alpha,\beta}{(x)} = \begin{cases} 
    x & x \not\in I_{\alpha} \land x \not \in I_{\beta}\\
    x - \min(I_{\alpha}) + \min(I_{\beta}), & x \in I_{\alpha},\\
    x - \min(I_{\beta}) + \min(I_{\alpha}), & x \in I_{\beta}\\
\end{cases}
\end{equation}

Outra definição equivalente seria:

\begin{equation}
 p_{\alpha,\beta}{(x)} = \begin{cases} 
    x & x \not\in I_{\alpha}\land x \not \in I_\beta\\
    x - \max(I_{\alpha}) + \max(I_{\beta}), & x \in I_{\alpha},\\
    x - \max(I_{\beta}) + \max(I_{\alpha}), & x \in I_{\beta}\\
\end{cases}
\end{equation}

Se $I_{\alpha} = I_{\beta}$ temos a permutação identidade de par de intervalos para espaços densos. É facilmente verificável que a fórmula se reduz à seguinte forma: $p_{\beta,\beta}{(x)} = p_{\alpha,\alpha}{(x)} = x$.

Repare que também para todo $\alpha,\beta \in \mathbb{N}$ temos $p_{\alpha,\beta} = p_{\beta, \alpha}$ e também que $p_{\alpha,\beta} \circ p_{\alpha,\beta}= \text{id}_M$.

\section{Função de permutação de espaços densos}

Com isso podemos definir a função de permutação de espaços densos $\sigma$ onde $\sigma: M \to M$ de forma que dada uma sequência de intervalos arbitrários $\{I_n\}_{n\in\mathbb{N}}$ e duas sequências de números naturais $\{a_n\}_{n\in\mathbb{N}}$ e $\{b_n\}_{n\in\mathbb{N}}$, temos então:

\begin{equation}
 \sigma = \prod_{n=1}^{\infty} p_{a_n,b_n}
\end{equation}

De forma intuitiva, a função pode ser tanto composta de infinitas permutações de pares de elementos (é um produtório de composição de funções) ou também pode ser definida de forma que $\exists m \in \mathbb{N}$ onde $\forall n > m : a_n=b_n$, também implicando que $p_{a_n,b_n}=id_M$, fazendo assim que seja composta de uma quantidade limitada de permutação de pares de intervalos. É claro que com esse fato podemos simplificar a equação para:

\begin{equation}
\sigma = \prod_{n=1}^{m} p_{a_n,b_n}
\end{equation}

\section{Exemplo}

Considere $M = \mathbb{R}$ (conjunto dos números reais)

\subsection{Permutação de par de intervalos}

Vamos definir dois intervalos:
\begin{itemize}
\item $I_1 = [0, 1]$
\item $I_2 = [2, 3]$
\end{itemize}

Agora, vamos aplicar a permutação $p_{1,2}(x)$:

\begin{equation}
p_{1,2}(x) = \begin{cases} 
    x & x \not\in [0, 1] \land x \not \in [2, 3]\\
    x - \min(I_1) + \min(I_2) = x + 2, & x \in [0, 1],\\
    x - \min(I_2) + \min(I_1) = x - 2, & x \in [2, 3]\\
\end{cases}
\end{equation}

Exemplos:
\begin{enumerate}
\item $p_{1,2}(0.5) = 0.5 + 2 = 2.5$
\item $p_{1,2}(2.5) = 2.5 - 2 = 0.5$
\item $p_{1,2}(4) = 4$ (não está em nenhum dos intervalos)
\end{enumerate}

\subsection{Função de permutação de espaços densos}

Vamos criar uma sequência finita de intervalos e duas sequências finitas de números naturais:

Intervalos: $(I_n)_{n=1}^3 = ([0, 1], [2, 3], [4, 5])$\\
Sequência a: $(a_n)_{n=1}^3 = (1, 2, 3)$\\
Sequência b: $(b_n)_{n=1}^3 = (2, 3, 1)$

Agora, definimos $\sigma = p_{1,2} \circ p_{2,3} \circ p_{3,1}$

Formalmente, podemos escrever:

\begin{equation}
\sigma = \prod_{n=1}^{3} p_{a_n,b_n} = p_{1,2} \circ p_{2,3} \circ p_{3,1}
\end{equation}

Vamos aplicar $\sigma$ a alguns pontos:

\begin{enumerate}
\item $\sigma(0.5):$
   \begin{itemize}
   \item $p_{3,1}(0.5) = 0.5$ (não está em $I_3$ nem em $I_1$)
   \item $p_{2,3}(0.5) = 0.5$ (não está em $I_2$ nem em $I_3$)
   \item $p_{1,2}(0.5) = 2.5$ (está em $I_1$)
   \end{itemize}
   Resultado: $\sigma(0.5) = 2.5$

\item $\sigma(2.5):$
   \begin{itemize}
   \item $p_{3,1}(2.5) = 2.5$ (não está em $I_3$ nem em $I_1$)
   \item $p_{2,3}(2.5) = 4.5$ (está em $I_2$)
   \item $p_{1,2}(4.5) = 4.5$ (não está em $I_1$ nem em $I_2$)
   \end{itemize}
   Resultado: $\sigma(2.5) = 4.5$

\item $\sigma(4.5):$
   \begin{itemize}
   \item $p_{3,1}(4.5) = 0.5$ (está em $I_3$)
   \item $p_{2,3}(0.5) = 0.5$ (não está em $I_2$ nem em $I_3$)
   \item $p_{1,2}(0.5) = 2.5$ (está em $I_1$)
   \end{itemize}
   Resultado: $\sigma(4.5) = 2.5$
\end{enumerate}

Observe que esta permutação $\sigma$ efetivamente "rotaciona" os elementos entre os três intervalos:
\begin{itemize}
\item Elementos de $I_1$ são movidos para $I_2$
\item Elementos de $I_2$ são movidos para $I_3$
\item Elementos de $I_3$ são movidos para $I_1$
\end{itemize}

Elementos fora desses intervalos permanecem inalterados, ou seja, $\forall x \not\in I_1 \cup I_2 \cup I_3, \sigma(x) = x$.

Podemos verificar que $\sigma$ é uma bijeção, pois cada elemento tem uma imagem única e todo elemento do conjunto é atingido pela função. Além disso, podemos observar que $\sigma \circ \sigma \circ \sigma = id_{\mathbb{R}}$, ou seja, aplicar $\sigma$ três vezes resulta na função identidade.

\end{document}