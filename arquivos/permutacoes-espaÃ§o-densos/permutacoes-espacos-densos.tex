\documentclass[12pt,a4paper]{article}
\usepackage[utf8]{inputenc}
\usepackage[T1]{fontenc}
\usepackage{amsmath,amssymb,amsfonts}
\usepackage[left=2cm,right=2cm,top=2cm,bottom=2cm]{geometry}
\usepackage{hyperref}

\title{Permutações em Espaços Densos}
\author{}
\date{}

\newtheorem{theorem}{Teorema}
\newtheorem{lemma}[theorem]{Lema}
\newtheorem{proposition}[theorem]{Proposição}
\newtheorem{corollary}[theorem]{Corolário}
\newtheorem{definition}{Definição}

\newenvironment{proof}[1][Prova]{\textbf{#1.} }{\ \rule{0.5em}{0.5em}}


\begin{document}

\maketitle

\section{Permutações em espaços densos}

Dado $(M, +, \leq)$ um monóide de adição, ordenado e denso de forma que para quaisquer $a,b \in M$ com $a<b$, existe $c \in M$ tal que $a < c < b$, definimos a operação $p$ de permutação de par de intervalos para espaços densos que é uma bijeção $p_{\alpha,\beta}: M \to M$ onde $x \in M$ e $\alpha,\beta \in \mathbb{N}$ e escolhemos 2 intervalos arbitrários disjuntos $I_{\alpha}$ e $I_{\beta}$ em que $\|I_{\alpha}\|=\|I_{\beta}\|$ ou seja eles tem mesmo tamanho, de forma que:

\begin{equation}
 p_{\alpha,\beta}{(x)} = \begin{cases} 
    x & x \not\in I_{\alpha} \land x \not \in I_{\beta}\\
    x - \min(I_{\alpha}) + \min(I_{\beta}), & x \in I_{\alpha},\\
    x - \min(I_{\beta}) + \min(I_{\alpha}), & x \in I_{\beta}\\
\end{cases}
\end{equation}

Outra definição equivalente seria:

\begin{equation}
 p_{\alpha,\beta}{(x)} = \begin{cases} 
    x & x \not\in I_{\alpha}\land x \not \in I_\beta\\
    x - \max(I_{\alpha}) + \max(I_{\beta}), & x \in I_{\alpha},\\
    x - \max(I_{\beta}) + \max(I_{\alpha}), & x \in I_{\beta}\\
\end{cases}
\end{equation}

Se $I_{\alpha} = I_{\beta}$ temos a permutação identidade de par de intervalos para espaços densos. É facilmente verificável que a fórmula se reduz à seguinte forma: $p_{\beta,\beta}{(x)} = p_{\alpha,\alpha}{(x)} = x$.

Repare que também para todo $\alpha,\beta \in \mathbb{N}$ temos $p_{\alpha,\beta} = p_{\beta, \alpha}$ e também que $p_{\alpha,\beta} \circ p_{\alpha,\beta}= \text{id}_M$.

\section{Função de permutação de espaços densos}

Com isso podemos definir a função de permutação de espaços densos $\sigma$ onde $\sigma: M \to M$ de forma que dada uma sequência de intervalos arbitrários disjuntos $\{I_n\}_{n\in\mathbb{N}}$ que tem o mesmo tamanho e duas sequências de números naturais $\{a_n\}_{n\in\mathbb{N}}$ e $\{b_n\}_{n\in\mathbb{N}}$, temos então:

\begin{equation}
 \sigma = \prod_{n=1}^{\infty} p_{a_n,b_n}
\end{equation}

De forma intuitiva, a função pode ser tanto composta de infinitas permutações de pares de elementos (é um produtório de composição de funções) ou também pode ser definida de forma que $\exists m \in \mathbb{N}$ onde $\forall n > m : a_n=b_n$, também implicando que $p_{a_n,b_n}=id_M$, fazendo assim que seja composta de uma quantidade limitada de permutação de pares de intervalos. É claro que com esse fato podemos simplificar a equação para:

\begin{equation}
\sigma = \prod_{n=1}^{m} p_{a_n,b_n}
\end{equation}

\section{Exemplo}

Considere $M = \mathbb{R}$ (conjunto dos números reais)

\subsection{Permutação de par de intervalos}

Vamos definir dois intervalos:
\begin{itemize}
\item $I_1 = [0, 1]$
\item $I_2 = [2, 3]$
\end{itemize}

Agora, vamos aplicar a permutação $p_{1,2}(x)$:

\begin{equation}
p_{1,2}(x) = \begin{cases} 
    x & x \not\in [0, 1] \land x \not \in [2, 3]\\
    x - \min(I_1) + \min(I_2) = x + 2, & x \in [0, 1],\\
    x - \min(I_2) + \min(I_1) = x - 2, & x \in [2, 3]\\
\end{cases}
\end{equation}

Exemplos:
\begin{enumerate}
\item $p_{1,2}(0.5) = 0.5 + 2 = 2.5$
\item $p_{1,2}(2.5) = 2.5 - 2 = 0.5$
\item $p_{1,2}(4) = 4$ (não está em nenhum dos intervalos)
\end{enumerate}

\subsection{Função de permutação de espaços densos}

Vamos criar uma sequência finita de intervalos e duas sequências finitas de números naturais:

Intervalos: $(I_n)_{n=1}^3 = ([0, 1], [2, 3], [4, 5])$\\
Sequência a: $(a_n)_{n=1}^3 = (1, 2, 3)$\\
Sequência b: $(b_n)_{n=1}^3 = (2, 3, 1)$

Agora, definimos $\sigma = p_{1,2} \circ p_{2,3} \circ p_{3,1}$

Formalmente, podemos escrever:

\begin{equation}
\sigma = \prod_{n=1}^{3} p_{a_n,b_n} = p_{1,2} \circ p_{2,3} \circ p_{3,1}
\end{equation}

Vamos aplicar $\sigma$ a alguns pontos:

\begin{enumerate}
\item $\sigma(0.5):$
   \begin{itemize}
   \item $p_{3,1}(0.5) = 0.5$ (não está em $I_3$ nem em $I_1$)
   \item $p_{2,3}(0.5) = 0.5$ (não está em $I_2$ nem em $I_3$)
   \item $p_{1,2}(0.5) = 2.5$ (está em $I_1$)
   \end{itemize}
   Resultado: $\sigma(0.5) = 2.5$

\item $\sigma(2.5):$
   \begin{itemize}
   \item $p_{3,1}(2.5) = 2.5$ (não está em $I_3$ nem em $I_1$)
   \item $p_{2,3}(2.5) = 4.5$ (está em $I_2$)
   \item $p_{1,2}(4.5) = 4.5$ (não está em $I_1$ nem em $I_2$)
   \end{itemize}
   Resultado: $\sigma(2.5) = 4.5$

\item $\sigma(4.5):$
   \begin{itemize}
   \item $p_{3,1}(4.5) = 0.5$ (está em $I_3$)
   \item $p_{2,3}(0.5) = 0.5$ (não está em $I_2$ nem em $I_3$)
   \item $p_{1,2}(0.5) = 2.5$ (está em $I_1$)
   \end{itemize}
   Resultado: $\sigma(4.5) = 2.5$
\end{enumerate}

Observe que esta permutação $\sigma$ efetivamente "rotaciona" os elementos entre os três intervalos:
\begin{itemize}
\item Elementos de $I_1$ são movidos para $I_2$
\item Elementos de $I_2$ são movidos para $I_3$
\item Elementos de $I_3$ são movidos para $I_1$
\end{itemize}

Elementos fora desses intervalos permanecem inalterados, ou seja, $\forall x \not\in I_1 \cup I_2 \cup I_3, \sigma(x) = x$.

Podemos verificar que $\sigma$ é uma bijeção, pois cada elemento tem uma imagem única e todo elemento do conjunto é atingido pela função. Além disso, podemos observar que $\sigma \circ \sigma \circ \sigma = id_{\mathbb{R}}$, ou seja, aplicar $\sigma$ três vezes resulta na função identidade.

\section{Função de permutação de pares de intervalos de espaços densos com inversão de intervalos}

Para maior flexibilidade para permutação de espaços densos é interessante também poder modificar e inverter a ordem
de forma que intervalos de $x$ crescentes possam se tornar descrescentes permutando a ordem desses intervalos, definimos um $B=\{\top, \bot\}$ e $s_1,s_2 \in B$ com isso podemos ajustar a função de forma que:

\begin{equation}
   {}^{(a,b)}p_{\alpha,\beta}{(x)} = \begin{cases} 
      x & x \not\in I_{\alpha} \land x \not \in I_{\beta}\\
      - x - \min(I_{\alpha}) + \max(I_{\beta}), & x \in I_{\alpha} \land s_1 \text{ é Verdadeiro},\\
      x - \min(I_{\alpha}) + \min(I_{\beta}), & x \in I_{\alpha} \land s_1 \text{ é Falso},\\
      - x - \min(I_{\beta}) + \max(I_{\alpha}), & x \in I_{\beta} \land s_2 \text{ é Verdadeiro},\\
      x - \min(I_{\beta}) + \min(I_{\alpha}), & x \in I_{\beta} \land s_2 \text{ é Falso}\\
  \end{cases}
\end{equation}

Repare nas seguinte propriedade ${}^{(\top,\bot)}p_{\alpha,\beta} \circ {}^{(\top,\bot)}p_{\alpha,\beta} = {}^{(\bot,\bot)}p_{\alpha,\beta} \circ {}^{(\top,\top)}p_{\alpha,\beta}$
que também pode ser generalizada de forma que para sequencia de booleanos $\{q_{1,n}\}_{n\in\mathbb{N}}$ e 
$\{q_{2,n}\}_{n\in\mathbb{N}}$ e $k \in\mathbb{N}$ e ${j_n = [1,2,1,2 \dots]}$ e ${k_n = [2,1,2,1 \dots]}$:

\begin{equation}
   \prod_{n=1}^{k} {}^{(q_{1,n}, q_{2,n})}p_{\alpha,\beta} = (\prod_{n=1}^{k-1} {}^{(\bot, \bot)}p_{\alpha,\beta}) \circ {}^{(\bigoplus q_{j_n,n}, \bigoplus q_{k_n,n})}p_{\alpha,\beta}
\end{equation}

Lembrar que para $k$ impar $\prod_{n=1}^{k-1} {}^{(\bot, \bot)}p_{\alpha,\beta} = \text{id}_M$ e para k par $\prod_{n=1}^{k-1} {}^{(\bot, \bot)}p_{\alpha,\beta} = {}^{(\bot, \bot)}p_{\alpha,\beta}$, como possivel simplificação
para essa propriedade generalizada. 
\vspace{0.5cm}

Com essa função ${}^{(a,b)}p_{\alpha,\beta}$ definimos uma sequencia de pares de booleanos $\{s_n\}_{n\in\mathbb{N}}$, de forma a definir nova função de permutação com inversão:

\begin{equation}
   \sigma^{*} = \prod_{n=1}^{\infty} {}^{s_n}p_{a_n,b_n}
\end{equation}

Seguindo a mesma lógica que anteriormente caso que se $\exists m \in \mathbb{N}$ onde $\forall n > m : a_n=b_n$, logo:

\begin{equation}
   \sigma^{*}  = \prod_{n=1}^{m} {}^{s_n}p_{a_n,b_n}
\end{equation}

\section{Função de permutação de pares de intervalos com intervalos de tamanho váriaveis.}

Umas das retrições que colocamos anteriormente era que ambos intervalos $I_{\alpha}$ e $I_{\beta}$ deveriam ter tamanho iguais, ou seja $\|I_{\alpha}\|=\|I_{\beta}\|$
assim preservando escala, umas das propriedades das equações (1) e (3) é que para uma função arbitrária $f$, temos:

\begin{equation}
   \int_{-\infty}^{\infty} f(x)\, dx = \int_{-\infty}^{\infty} f(p_{\alpha, \beta}(x))\, dx = \int_{-\infty}^{\infty} f(\sigma(x))\, dx
\end{equation}

Nós podemos fazer uma variação da equação que não satisfaz essa própriedade e não preserva a escala, porém possibilita
 intervalos $I_{\alpha}$ e $I_{\beta}$ de tamanhos diferentes, com isso é necessário "alongar" ou "diminuir" esses intervalos, 
 sendo assim há necessidade também de um objeto que possibilite a multiplicação por escalares.

 \subsection{Nova definição para intervalos de tamanho váriaveis e com inversão de intervalos}

 Dado $(A, +, *, \leq)$ um campo em $A$ que é ordenado e denso de forma que para quaisquer
  $a,b \in M$ com $a<b$, existe $c \in A$ tal que $a < c < b$, definimos a operação
   $v$ de permutação de par de intervalos váriaveis para espaços densos que é uma 
   bijeção ${}^{(s_1,s_2)}v_{\alpha,\beta}: M \to M$ onde 
   $x \in M$ e $\alpha,\beta \in \mathbb{N}$ e escolhemos 2 intervalos arbitrários 
   disjuntos $I_{\alpha}$ e $I_{\beta}$ e definimos um $B=\{\top, \bot\}$ e 
   $s_1,s_2 \in B$, de forma que:

 \begin{equation}
   {}^{(a,b)}v_{\alpha,\beta}{(x)} = \begin{cases} 
      x & x \not\in I_{\alpha} \land x \not \in I_{\beta}\\
      - \min(I_{\alpha}) + \max(I_{\beta}) - x\frac{\|I_{\beta}\|}{\|I_{\alpha}\|}, & x \in I_{\alpha} \land s_1 \text{ é Verdadeiro},\\
      x\frac{\|I_{\beta}\|}{\|I_{\alpha}\|} - \min(I_{\alpha}) + \min(I_{\beta}), & x \in I_{\alpha} \land s_1 \text{ é Falso},\\
      - \min(I_{\beta}) + \max(I_{\alpha}) - x\frac{\|I_{\alpha}\|}{\|I_{\beta}\|}, & x \in I_{\beta} \land s_2 \text{ é Verdadeiro},\\
      x\frac{\|I_{\alpha}\|}{\|I_{\beta}\|} - \min(I_{\beta}) + \min(I_{\alpha}), & x \in I_{\beta} \land s_2 \text{ é Falso}\\
  \end{cases}
\end{equation}

Da mesma forma temos $\sigma^{**}$ que não necessáriamente preserva a escala.

\begin{equation}
   \sigma^{**} = \prod_{n=1}^{\infty} {}^{s_n}v_{a_n,b_n}
\end{equation}

E caso a sequencia $\{I_n\}_{n\in\mathbb{N}}$ ter intervalos de tamanhos diferentes, logo:

\begin{equation}
   \int_{-\infty}^{\infty} f(x)\, dx \neq \int_{-\infty}^{\infty} f(\sigma^{**}(x))\, dx
\end{equation}


\section{Teorema do Ponto Fixo}

Para qualquer $\sigma^*$, existe pelo menos um ponto $x \in M$ tal que $\sigma^*(x) = x$. Este ponto fixo pode ser encontrado nos pontos de fronteira dos intervalos ou nos pontos que não pertencem a nenhum intervalo de uma sequência finita $\{I_n\}_{n\in\mathbb{N}}$.

\begin{theorem}[Ponto Fixo]
Para qualquer $\sigma^*$, existe pelo menos um ponto $x \in M$ tal que $\sigma^*(x) = x$.
\end{theorem}

\begin{proof}
Consideremos duas possibilidades:

1) Se existe algum $x \in M$ que não pertence a nenhum intervalo da sequência $\{I_n\}_{n\in\mathbb{N}}$, então pela definição de $\sigma^*$, temos $\sigma^*(x) = x$, e o teorema está provado.

2) Caso contrário, todos os pontos de $M$ estão em algum intervalo da sequência. Neste caso, consideremos os pontos de fronteira dos intervalos. Seja $x = \min(I_k)$ para algum $k$. Temos duas subcasos:

   a) Se $x$ é mapeado para outro intervalo $I_j$, então $\sigma^*(\max(I_j)) = x$, pois $\sigma^*$ preserva a ordem dentro dos intervalos. Portanto, $\max(I_j)$ é um ponto fixo.
   
   b) Se $x$ é mapeado para si mesmo, então x é um ponto fixo.

Em todos os casos, encontramos um ponto fixo, o que prova o teorema.
\end{proof}


\section{Propriedade de Aproximação}

Para qualquer função contínua e bijetiva $f: A \to A$ e $\epsilon > 0$, existe uma função de permutação $\sigma^{**}$ tal que:

\begin{equation}
\sup_{x \in M} d(f(x), \sigma^{**}(x)) < \epsilon
\end{equation}

onde $d$ é uma métrica adequada em $A$. Esta propriedade sugere que as funções de permutação podem aproximar arbitrariamente bem qualquer função contínua e bijetiva no espaço.

\begin{theorem}[Aproximação]
Para qualquer função contínua e bijetiva $f: A \to A$ e $\epsilon > 0$, existe uma função de permutação $\sigma^{**}$ tal que:
\[\sup_{x \in M} d(f(x), \sigma^{**}(x)) < \epsilon\]
onde $d$ é uma métrica adequada em $A$.
\end{theorem}

\begin{proof}[Esboço da prova]
A ideia principal é usar a densidade de $A$ e a continuidade de $f$ para construir $\sigma^{**}$:

1) Como $A$ é denso, podemos escolher uma sequência de pontos $\{x_n\}_{n\in\mathbb{N}}$ que é $\epsilon/2$-densa em $A$.

2) Para cada $x_n$, escolhemos intervalos $I_n$ e $J_n$ tais que:
   \begin{itemize}
   \item $x_n \in I_n$
   \item $f(x_n) \in J_n$
   \item $\text{diam}(I_n) < \epsilon/2$ e $\text{diam}(J_n) < \epsilon/2$
   \end{itemize}

3) Definimos $\sigma^{**}$ para mapear $I_n$ em $J_n$ de maneira linear (ou preservando a ordem se estivermos trabalhando em um espaço mais geral).

4) Para pontos fora dos intervalos escolhidos, definimos $\sigma^{**}$ de forma a ser uma bijeção (isso é possível porque $f$ é bijetiva).

5) Por construção, para qualquer $x \in A$, existe um $x_n$ tal que $d(x, x_n) < \epsilon/2$. Então:
   \[d(f(x), \sigma^{**}(x)) \leq d(f(x), f(x_n)) + d(f(x_n), \sigma^{**}(x_n)) + d(\sigma^{**}(x_n), \sigma^{**}(x))\]

   Cada termo nesta soma é menor que $\epsilon/3$ (assumindo que escolhemos $\epsilon$ suficientemente pequeno na construção), o que completa a prova.
\end{proof}
   

\section{Entropia Topológica}

Podemos definir a entropia topológica $h(\sigma)$ para a função de permutação $\sigma$ como:

\begin{equation}
h(\sigma) = \lim_{n \to \infty} \frac{1}{n} \log N(n, \epsilon)
\end{equation}

onde $N(n, \epsilon)$ é o número mínimo de conjuntos de diâmetro menor que $\epsilon$ necessários para cobrir o conjunto $\{x, \sigma(x), \sigma^2(x), ..., \sigma^{n-1}(x)\}$ para algum $x \in M$. Esta medida quantifica a complexidade dinâmica da permutação.

\begin{theorem}
A entropia topológica de $\sigma$ é finita se e somente se $\sigma$ é composta de um número finito de permutações de intervalos.
\end{theorem}

\begin{proof}[Esboço da prova]
1) Se $\sigma$ é composta de um número finito $k$ de permutações de intervalos, então para qualquer $x \in M$, o conjunto $\{x, \sigma(x), \sigma^2(x), ..., \sigma^{n-1}(x)\}$ tem no máximo $k$ elementos distintos. Portanto, $N(n, \epsilon)$ é limitado por uma constante independente de $n$, e $h(\sigma) = 0$.

2) Se $\sigma$ é composta de um número infinito de permutações de intervalos, podemos construir um ponto $x \in M$ tal que sua órbita $\{x, \sigma(x), \sigma^2(x), ...\}$ é infinita e separada (isto é, existe um $\delta > 0$ tal que $d(\sigma^i(x), \sigma^j(x)) > \delta$ para $i \neq j$). Neste caso, $N(n, \epsilon)$ cresce exponencialmente com $n$ para $\epsilon < \delta/2$, o que implica que $h(\sigma) > 0$.
\end{proof}
   


\end{document}